\begin{usage}
Assignment 2, Year 2004, Fall Term, School of Computer Science at the University of Waterloo.
\end{usage}

When using the {\tt long} primitive datatype in Java, care must be taken
to ensure that the result of an arithmetic operation such as $+$ or $\ast$
lies in the range
%$$[-2^{63}\ldots 2^{63}-1
$$-9223372036854775808\ldots 9223372036854775807.$$
If we have a modulus $p$
that satisfies $2\leq p \leq 3037000500$, and we represent integers modulo
$p$ in the range $0\ldots p-1$, then we can perform arithmetic modulo
$p$ using the {\tt long} datatype.  The upper limit for $p$ is carefully
chosen so that $(p-1)^2 \leq 9223372036854775807$.  If $a$, $b$
and $p$ satisfy the conditions $0 \leq a <
p$, $0 \leq b < p$ and $2 \leq p \leq 3037000500$, then
addition, multiplication and negation modulo $p$
can be performed in Java, using variables of type {\tt long},
as \verb[(a + b) % p[, \verb[(a * b) % p[ and \verb[p - a[ respectively.

\medskip

Consider the following method for computing the power of integers
modulo $p$.
\begin{verbatim}
public static long power(long x, long k, long p) {
// pre: 0 <= x < p,  k > 0,  2 <= p <= 3037000500
// post: returns x^k modulo p
//       the return value will be in the range [0..p-1]
{
   long a = 1;
   for (long i = 1; i <= k; i++)
      a = (a * x) % p;
   return a;
}
\end{verbatim}
The method is correct, but has a running time proportional to $\Theta(k)$.

\medskip

Your task is to provide an alternative implementation of method
{\tt power}.  Your method should have running time proportional to
$\Theta(\log k)$.
%
Like the naive implementation given above, you may only use control
statements like {\tt if} and {\tt while}, together with the Java
operators \verb[+, -, *, /[.


\begin{hint}
%Note that $c^{1041} = c^{1024}c^{16}c$.
How can you compute $c^{16}$ using only four instead of fifteen
multiplications?  How do the binary expansions of $k$ and $k/2$ compare?
\end{hint}

\begin{solution}
   The body of {\tt power} can be replaced with:
\begin{verbatim}
   long a = 1;
   while (k != 0) {
      if (2*(k/2) != k) a = (a*x) % p;
      x = (x*x) % p;
      k = k/2;
   }
   return a;
\end{verbatim}


 A polylogarithmically bounded solution (i.e., $(\lg k)^c$
 for some constant $c$) would also be considered for part marks.

 For example, here's a $\Theta( (\lg k)^2 )$ solution:

\begin{verbatim}
    long tracker = 1;
    long storage = x;
    if (k==0) {return 0;}

    while ( tracker * 2 <= k && tracker * 2 > 0 )
    { tracker = tracker * 2;
      storage = (storage*storage) %p;
    }
    storage = storage * power(x,k-tracker,p);
    return storage % p;
\end{verbatim}

\end{solution}

\begin{markingScheme}
  4 marks for a $\Theta(\lg k)$ solution

  3 marks for a polylogarithmically bounded solution (i.e., $(\lg k)^c$
 for some constant $c$)

  2 and 1 marks by discretion (e.g., correct verbal description?)
\end{markingScheme}
