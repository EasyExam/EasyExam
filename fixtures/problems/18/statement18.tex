\begin{authorship}
Cormen, Leiserson, Rivest \& Stein, Problem 6-3, with modifications by T. Biedl
\end{authorship}
\begin{usage}
\begin{itemize}
\item Fall 2005, School of Computer Science at the University of Waterloo, Assignment 2 CS240.
\item Spring 2006, University of Waterloo, SCS, Assignment 2 CS240
\begin{description}
\item[maxpoints] 12.00
\item[max] 12.00
\item[median] 6.25
\item[average] 6.02
\end{description}


\end{itemize}
\end{usage}
A {\em 2-way-ordered-matrix} (or 2WOM) is an
$m\times n$-matrix of integers (with $m\geq 1$
and $n\geq 1$) for which the entries of each row are sorted in decreasing
order from left to right, and the entries in each column are sorted in
decreasing order from top to bottom.  In order to be able to store less
than $mn$ integers, we allow that some of the entries in the matrix are
$-\infty$.  However, these $-\infty$ elements must also adhere to the ordering
of rows and columns, i.e., they must be at the end of a row and at the
bottom of a column.

\begin{codeExample}
  An example of a 2WOM is given below.
$$\left( \begin{array}{cccc}
    15 & 12 & 10 & 9 \\
    14 & 11 & 8 & 7 \\
    13 & 6 & 2 & 1 \\
    5 & 3 & -\infty & -\infty \\
    4 & -\infty & -\infty & -\infty \\
  \end{array}
\right) $$
\end{codeExample}
\begin{enumerate}
\item
Show how to use a 2WOM to implement the ADT PriorityQueue. For each
algorithm you provide, justify its correctness. You should achieve the
following running times for the methods:
\begin{itemize}
%\item[(i)] You should find the maximum element in $O(1)$ time.
\item[(i)] You should extract the maximum element in $O(m+n)$ time for
    an $m\times n$ 2WOM.
\item[(ii)] You should insert a new element in $O(m+n)$ time for a
    an $m\times n$ 2WOM that contains at least
    one entry that is $-\infty$. \\
    (You don't need to consider the case when the 2WOM is full.)
\end{itemize}

%\begin{markingScheme}
%20 marks: 5 for the correct algorithm for (i), 5 for correct
%algorithm for (ii).  Both algorithms should have an explanation
%with them, but give up to 10 marks for the better of the two.
%Their correctness needn't be as detailed as mine, but they should
%make some attempt of explaining why they indeed get a 2WOM.
%\end{markingScheme}


\begin{solution}
The data structure we designed in this question is called
\emph{Young tableaux}. The structure itself is much older
than computers. It was introduced in 1900 and since
then it played an important role in theory of symmetric
groups.

\begin{itemize}
\item[(i)]
    The maximum element is stored at location (1,1) in the matrix,
    so we can find it in $O(1)$ time.  However, if we simply took it out
    we'd leave a hole in the matrix.
    So we replace the maximum
    with $-\infty$. This will violate the 2WOM property, thus we need
    to design an algorithm that restores the 2WOM property.

    The algorithm will work in a similar way as Heapify for heaps.
    It will compare the element that violates the 2WOM property with
    its neighbors to the right and down and swap this element with
    the larger of the two. This process is repeated until the 2WOM
    property is restored. This procedure can be summarized as follows
    (we run Fix2WOMProperty(1,1) after inserting the element $-\infty$
    at position $(1,1)$):

    \begin{verbatim}
    Fix2WOMProperty(i,j):
      // pre: M satisfies the 2WOM property except for location (i,j)
      //      the element at x = M(i,j) can be replaced by x'>=x such that
      //      the resulting matrix would satisfy the 2WOM property
      // post: M satisfies the 2WOM property

      x = M(i,j)
      y = M(i+1,j) (-infty if i==n)
      z = M(i,j+1) (-infty if j==m)

      while (x < max{y,z}) {
        // inv: x = M(i,j), y = M(i+1,j), z = M(i,j+1)
        //      x violates the 2WOM property
        //      x could be replaced by x'>=x such that the
        //      resulting matrix M would satisfy the 2WOM property

        if (z < y) {
            M(i,j) = y, M(i+1,j) = x
            i = i+1
        else {
            M(i,j) = z, M(i,j+1) = x
            j = j+1
        }

        y = M(i+1,j) (-infty if i==n)
        z = M(i,j+1) (-infty if j==m)

        // inv: x = M(i,j), y = M(i+1,j), z = M(i,j+1)
        //      x could be replaced by x'>x such that
        //      resulting matrix M would satisfy 2WOM property
      }
    \end{verbatim}

    We need to argue that this procedure indeed creates a 2WOM.  We
    will do that by considering the invariant outlined in the pseudocode.
    The pre-condition of the code is satisfied upon entering the function
    (M(1,1) could be replaced by $\infty$ which would satisfy the 2WOM
    condition).  The precondition clearly guarantees that the
    invariant is satisfied upon entering the first iteration.

    We need to show that even after the iteration is finished, the
    invariant is still satisfied.  Let $u=M(i-1,j)$ and $v=M(i,j-1)$
    (or $\infty$, if we are on the left or on the top boundary of the
    matrix).  Let $x'\geq x$ be the element that we could replace $x$
    with (in $M$ when calling the routine) and get a 2WOM.  In
    picture:

    $$\left( \begin{array}{ccc}
      ? & u & ? \\
      v & x' & z \\
      ? & y & ? \\
    \end{array}
    \right)
    $$

    We hence have $u\geq x' \geq y$ and $v\geq x'\geq z$ and know $x'\geq x$.
    We distinguish cases:
    \begin{itemize}
    \item Assume $y\geq z$.  We exchange $y$ and $x$, so now $y$ is
      in position $(i,j)$. Since $v\geq x' \geq y \geq z$, now row $i$ is sorted in decreasing order.
      Also, $u\geq x' \geq y > x$, so column $j$ is sorted in decreasing order,
      except that $x$ need not be larger than the neighbour underneath it.
      However, if we replaced value $x$ by value $y$ then column $j$ would be
      in decreasing order, so the invariant at the end of the iteration is satisfied.

    \item Assume $z\geq y$.  We then exchange $z$ and $x$, so now $z$
      is in position $(i,j)$.  Similarly as above now column $j$ is sorted in
      decreasing order, and row $i$ would be in decreasing order if we replaced
      value $x$ by value $z$. So the invariant at the end of the iteration is
      satisfied.
    \end{itemize}

    At the end of the iteration, if $x$ violates the 2WOM property, then
    we continue onto the next iteration and the invariant at the beginning of the
    iteration is automatically satisfied. If $x$ does not violate the 2WOM property,
    then we constructed a valid 2WOM.

    Note that each iteration of the Fix2WOMProperty increases either $i$ or $j$,
    so in total there can be at most $O(n+m)$ iterations, which take $O(n+m)$
    time total.

\item[(ii)]  We first need to find a place where a $-\infty$ is stored.
    Since we know that one exists, there must be one at location $(m,n)$,
    so we temporarily put the new element there.

    Now we fix up the 2WOM property, similar as during sift-up for heaps.
    However, we have to be careful now: for heaps we had only one parent
    (so no question as to who to exchange with if the new element is bigger),
    whereas now we must contemplate both the neighbour above and to the left,
    and exchange with the one that is smaller.

    The code to do this is actually almost identical to the one for part (ii),
    but we go up and left instead of down and right.  Thus, we call
    Fix2WOMPropertyReverse($m,n$) defined as follows:

    \begin{verbatim}
    Fix2WOMPropertyReverse(i,j):
      // pre: M satisfies the 2WOM property except for location $(i,j)$
      //      element at x = M(i,j) can be replaced by x'<=x such that
      //      resulting matrix would satisfy 2WOM property
      // post: M satisfies the 2WOM property

      x = M(i,j)
      y = M(i-1,j) (-infty if i==1)
      z = M(i,j-1) (-infty if j==1)

      while (x > min{y,z}) {
        // inv: x = M(i,j), y = M(i-1,j), z = M(i,j-1)
        //      x violates the 2WOM property
        //      x could be replaced by x'<=x such that
        //      resulting matrix M would satisfy 2WOM property

        if (z > y) {
            M(i,j) = y, M(i-1,j) = x
            i = i-1
        else {
            M(i,j) = z, M(i,j-1) = x
            j = j-1
        }

        y = M(i-1,j) (-infty if i==1)
        z = M(i,j-1) (-infty if j==1)

        // inv: x = M(i,j), y = M(i-1,j), z = M(i,j-1)
        //      x could be replaced by x'<=x such that
        //      resulting matrix M would satisfy 2WOM property
      }
    \end{verbatim}

    The correctness and time complexity of this procedure is argued
    exactly as in part (ii).
\end{itemize}
\end{solution}



\item Recall PriorityQueue sorting, which consists of inserting
$N$ elements into a PriorityQueue, and then extracting the maximum
$N$ times.  If you use the above implementation
of PriorityQueue, what asymptotic
running time can be achieved for $N$ elements?
Your answer should depend only on $N$, so in particular you should
specify how to choose the dimensions of the 2WOM.

\begin{solution}
I would choose $n=m=\lceil \sqrt{N} \rceil$; then the 2WOM is guaranteed
to have enough space for all numbers.  Now insert and extractMax each
take $O(n+m)=O(\sqrt{N})$ time, so the total time is
$O(N\sqrt{N}) = O(N^{3/2})$.
\end{solution}

%\begin{markingScheme}
%2 marks for actually saying how to choose $n$ and $m$, 3 marks for
%analysing the run-time.
%\end{markingScheme}

\item
Implementing a PriorityQueue with a 2WOM has the advantage that some
other operations can be executed fairly efficiently.  In particular,
show how to find the minimum element that is not $-\infty$
    in an $m\times n$ 2WOM in $O(m+n)$ time.

\begin{solution}
The minimum element surely must have $-\infty$ (or no element at all) both
below it and to its right.  To find it, we have to search all such elements
and return the minimum among them.

To find these elements, start at location $(1,1)$ and go down to the last
element that is not
$-\infty$.  Then go to the right to the last element that is not
$-\infty$.  This is the current minimum, and we took $O(m+n)$ time to
get to it.  Any value to the right or below the current minimum is $-\infty$,
so to find other candidates, go to the row
above it and then to the right to the last element that is not $-\infty$.
The larger between this value and the current minimum becomes the new
current minimum.  Repeat this process, i.e., go up one row and then to the
right until the last element that is not $-\infty$, then compare with
the current minimum.  Repeat until we are in the first row or the last
column.  Since we never go down or left (after the initial
movement), we do at most $O(m+n)$ steps to find all elements that could be
a minimum, and hence spend $O(m+n)$ time.
\end{solution}

%\begin{markingScheme}
%10 marks total.  5 marks for the algorithm, 5 marks for explanation.
%(The explanation given above is at about the level of formality I would
%expect.)
%\end{markingScheme}


\end{enumerate}


\begin{suggestedMarkingScheme}
\begin{authorship}
Zhuliang Chen
\end{authorship}
For a total of 12 marks:
\begin{itemize}
  \item[(a)] 6 marks. 2 marks for the correct algorithm for (i), 2 marks for
       the correct algorithm for (ii). Both algorithms should have
       an explanation with them, but give up to 2 marks for the better
       of two.
  \item[(b)] 2 marks, 1 mark for actually saying how to choose n and m, 1
       mark for analyzing the running time.
  \item[(c)] 4 marks, 2 marks for the algorithm, 2 marks for explanation.
\end{itemize}
\textbf{ Students should not be penalized for only providing the
detailed algorithm, instead of detailed pseudocode.}
\end{suggestedMarkingScheme}
