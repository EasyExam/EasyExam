\begin{usage}
Assignment 1, Year 2004, Fall Term, School of Computer Science at the University of Waterloo.
Midterm, Year 2005, Winter Term, School of Computer Science at the University of Waterloo.
\end{usage}
%Relative asymptotic growths
\begin{authorship}
CLRS 3-2.
\end{authorship}
Fill in the table below to indicate
for each pair of expressions (A,B)
whether A is $O$, $o$, $\Omega$, or $\Theta$ of B.
Use ``Y'' for ``Yes'' and ``N'' for ``No''.
%
You do not need to justify your answers.
%
Assume that $k \geq 1$, $\epsilon > 0$, and $c > 1$ are
constants.
%
\begin{spaceForAnswer}
\renewcommand{\arraystretch}{2}
\begin{textOnly}
$$
\begin{array}{|c|c|c|c|c|c|}\hline
  A          &	B               &   \ \ \ O\ \ \    &   \ \ \ o\ \ \    & \ \ \ \Omega\ \ \ &\ \ \ \Theta\ \ \  \\ \hline
 \log^{k}n     &n^{\varepsilon }    	&    	&  	&  	&  	\\ \hline
 n^{k}       &	c^{n}              	&   	&  	&  	&  	\\ \hline
 \sqrt{n}    &	n ^{\sin n}        	&   	&  	&  	&  	\\ \hline
 2^{n}       &	2^{n/2}            	&   	&   	&  	&  	\\ \hline
 n^{\log c}   &	c^{\log n}          	&   	&  	&  	&  	\\ \hline
\end{array}
$$
\end{textOnly}
\end{spaceForAnswer}
\begin{solution}
\begin{authorship}J\'er\'emy Barbay\end{authorship}
$$
\begin{array}{|c|c|c|c|c|c|c|}\hline
  A          &	B               &   O   &   o   & \Omega&\theta \\	\hline
 \log^{k}n     &n^{\varepsilon }  &   Y	&   Y 	&  N	&  N	\\	\hline
 n^{k}       &	c^{n}           &   Y	&   Y	&  N	&  N	\\	\hline
 \sqrt{n}    &	n ^{\sin n}     &   N	&   N	&  N	&  N	\\	\hline
 2^{n}       &	2^{n/2}         &   N 	&   N	&  Y 	&  N	\\	\hline
 n^{\log c}   &	c^{\log n}       &   Y	&   N	&  Y	&  Y	\\	\hline
\end{array}
$$
\end{solution}
\begin{markingScheme}
3 marks for each line.  Deduct 1 for each wrong answer (to a max of
3 deductions per line).
\end{markingScheme}
