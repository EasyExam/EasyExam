\begin{authorship}Unknown\end{authorship}
\begin{usage}
Assignment 2, Year 2004, Fall Term, School of Computer Science at the University of Waterloo.
\end{usage}

%Assume the scenario on page 7-12 of the course notes.
\begin{enumerate}

\item Let $C_{\mbox{\scriptsize WORST}}$ be the expected number of
comparisons for a successful search given the worst possible ordering
of the keys.
%
Express $C_{\mbox{\scriptsize
WORST}}$ in terms of $n$ and $C_{\mbox{\scriptsize OPT}}$.

\item The (not so useful) Move-to-Back heuristic is the same as the
Move-to-Front heuristic but the most recently retrieved item is
swapped with the last item in the dictionary.
%
What is the expected cost of performing a successful search under the
Move-to-Back heuristic for the following probability distribution:
$p_i = 4/(3n)$ for $i=1\ldots n/2$ and $p_i = 2/(3n)$ for
$i=n/2+1\ldots n$.
%
Assume $n$ is even and give an exact answer in terms of $n$.
%
\begin{hint}
The correct solution has the form $c_1+c_2n$ for rational numbers
$c_1$ and $c_2$.
\end{hint}

\end{enumerate}

\begin{solution}
\begin{enumerate}

  \item Consider that $C_{\hbox{\scriptsize OPT}}$ = $\sum_{j=1}^{n} j  P_j$.
%
  Similarly, $C_{\hbox{\scriptsize WORST}}$ = $\sum_{j=1}^{n} (n - j + 1)P_j$
  which simplifies to $n+1-C_{\hbox{\scriptsize OPT}}$.


  \item Two solutions are possible. Both are worth full marks:
     \begin{enumerate}
	\item The first solution uses the provided hint. We substitute and evaluate
	the expected cost for two values of $n$ and then solve the system of equations.

	The probability that $k_i$ is before $k_j$ is $p_j/(p_i+p_j)$.
	Thus, the expected number
	of keys before $k_j$ is $\sum_{i,i\neq j} p_j/(p_i+p_j)$.

	Calculating for $n=2$,
	 %
	 %
	 \begin{eqnarray*}
	 C_{\hbox{\scriptsize MTB}} &= &
	  \sum_j p_j(1+ \sum_{i,i\neq j} \frac{p_j}{p_i+p_j})\\
	  & = & 1 + \sum_j p_j^2 \sum_{i,i\neq j} \frac{1}{p_i+p_j}\\
	  & = & 1 + \left (\frac{2}{3}\right )^2 \left (\frac{1}{1/3 +
                     2/3}\right )
   		  + \left (\frac{1}{3}\right )^2 \left (\frac{1}{2/3 + 1/3}
                    \right )   \\
	  & = & \frac{14}{9}
	 \end{eqnarray*}

	Similarly, for $n=4$ we get
	 %
	 %
	 \begin{eqnarray*}
	 C_{\hbox{\scriptsize MTB}} &= &
	  \sum_j p_j\left (1+ \sum_{i,i\neq j} \frac{p_j}{p_i+p_j}\right)\\
	  & = & 1 + \sum_j p_j^2 \sum_{i,i\neq j} \frac{1}{p_i+p_j}\\
	  & = & 1 + 2\left (\frac{1}{3}\right )^2 \left (\frac{1}{1/3 + 1/3} + \frac{2}{1/6 + 1/3}\right) \\
          & + &	2\left (\frac{1}{6}\right )^2 \left(\frac{2}{1/3 + 1/6} + \frac{1}{1/6 + 1/6}\right)   \\
	  & = & \frac{47}{18}
	 \end{eqnarray*}

	 Which gives us the linear system $c_1 + c_2(4) = 47/18$ and
 $c_1 + c_2(2) = 14/9$
	 which we can solve to get the following equation for expected cost:
	  $C_{\hbox{\scriptsize MTB}} = 1/2 + 19n/36$.

	\item The second solution (for the more intrepid) finds the solution more directly
	by solving the sums.

	 Let $x=4/(3n)$ and $y=2/(3n)$. Then
	 %
	 %
	 \begin{eqnarray*}
	 C_{\hbox{\scriptsize MTB}} &= &
	  \sum_j p_j(1+ \sum_{i,i\neq j} \frac{p_j}{p_i+p_j})\\
	  & = & 1 + \sum_j p_j^2 \sum_{i,i\neq j} \frac{1}{p_i+p_j}\\
	  & = & 1 + \sum_{j=1}^{n/2}x^2 \sum_{i,i\neq j} \frac{1}{p_i+x} + \sum_{j=\frac{n}{2} + 1}^{n}y^2 \sum_{i,i\neq j} \frac{1}{p_i+y} \\
	  & = & 1 + x^2 \sum_{j}^{n/2} \left ( \left (\frac{n}{2} - 1\right)
    \left (\frac{1}{x+x}\right) + \left(\frac{n}{2}\right )
   \left (\frac{1}{y+x}\right ) \right ) \\
	  & + &   + y^2 \sum_{n/2 + 1}^{n}  \left ( \left (\frac{n}{2} - 1
       \right ) \left (\frac{1}{y+y}\right ) + \left (\frac{n}{2}\right )
    \left (\frac{1}{x+y} \right ) \right ) \\
	  & = & 1 + \left (\frac{n}{2} \right )
	 \left (
	 \left (\frac{n}{2} \right)
	 \left (\frac{x^2+y^2}{x+y}\right ) +
	 \left (\frac{n}{2} -1\right ) \left (x^2\left( \frac{1}{x+x} \right )
	 + y^2\left(\frac{1}{y+y}\right) \right )
	 \right ) \\
	  & = & \frac{1}{2} + \frac{19n}{36}
	 \end{eqnarray*}
     \end{enumerate}
  \end{enumerate}
\end{solution}

\begin{INUTILE}
  \begin{markingScheme}
    \begin{enumerate}

    \item A correct answer is worth 2 marks. Justification is nice,
      but not required due to the wording of the question.  Assign 1
      mark for solutions that have simple arithmetic/summation errors
      but have the correct idea.

    \item 3 marks for a complete solution using either approach 2
      marks for a fairly complete solution with minor errors
      (arithmetic/summation) or inadequate justification 1 marks for
      an incomplete solution that has some of the correct ideas

    \end{enumerate}
\end{markingScheme}
\end{INUTILE}
