\begin{authorship}J{\'e}r{\'e}my Barbay, a simplified version of ternaryTreeProperties01\end{authorship}
\begin{usage}Assignment 2, CS240, Winter term 2005, University of Waterloo, Canada\end{usage}

A {\em ternary tree}, which we sometimes call a {\em general ternary tree},
is a tree such that each node has at most $3$ children.
A {\em proper} ternary tree is such that each internal node has exactly $3$ children.
%
Given a ternary tree $T$, let
  $i$ be the number of internal nodes,
  $e$ be the number of external nodes, and
  $h$ be the height.


\begin{markingScheme}
$6$ marks for the first question,
and $2$ marks for each of the other questions.
\end{markingScheme}


\begin{enumerate}

%%%%%%%%%%%%%%%%%%%%%%%%%%%%%%%%%%%%%%%%%%%%%%%%%%%%%%%%%%%%%%%%%%%%%%%%%%%%%%
\item Prove by induction that for any ternary tree,
%
$$h\leq i \leq {3^h-1\over 2}.$$
%
Note that this implies that the property is also true for proper
ternary trees.
%%%%%%%%%%%%%%%%%%%%%%%%%%%%%%%%%%%%%%%%%%%%%%%%%%%%%%%%%%%%%%%%%%%%%%%%%%%%%%
\begin{solution}

We prove this property by induction on $h$.
\begin{itemize}

\item {\bf The induction is based on the variable} $h$, the height of
the tree.

\item {\bf The induction hypothesis is} $H(h)=$ ``For any
ternary tree of height at most $h$ and with $i$ internal nodes,
%
$h\leq i \leq {3^h-1\over2}$''.

\item {\bf The base case of the hypothesis is} $H(0)=$''For any
ternary tree of height $0$ and with $i$ internal nodes,
$0\leq{i}\leq {3^0-1\over2}=0$''.
%
As a ternary tree of height zero can have only one external node and
no internal nodes ($i=0$), $H(0)$ {\bf is verified}.

\item For any value of $h\in\{1,\ldots,\infty\}$, {\bf suppose that
$H(h-1)$ is true.}  The inductive step goes as follow:
%
For any ternary tree $T$ of height $h\geq1$, consider the children of
the root of $T$: there are at least one, and at most three, and each
is of height at most $h-1$: $H(h-1)$ apply to them.
%
One one hand, $T$ has at least one sub-tree and its number of internal
nodes is at least the number of internal nodes (by induction
hypothesis at least $h-1$) of the tallest of its subtrees plus one:
$i\geq h-1+1=h$.
%
On the other hand, the number of internal nodes in each subtree is at
most ${3^{h-1}-1\over 2}$, and $T$ has at most $3$ subtrees, hence its
number of internal nodes is at most
$1+3*{3^{h-1}-1\over 2}
= {3^h-3+2\over2}
= {3^h-1\over2}$.
%
Hence the final inequality $h\leq i \leq {3^{h}-1\over 2}$, {\bf which
proves that $H(h)$ is verified}.

\item {\bf The conclusion} is that for $h\in\{0,\ldots,\infty\}$, any
ternary tree of height $h$ has between $h$ and ${3^h-1}\over2$ internal nodes.

\end{itemize}

\begin{markingScheme}
\begin{itemize}
\item $2$ mark if among the five points of the induction none is
missing, even if the induction is wrong (see the hint at the end of
the problem);
\item $4$ marks in general for the induction proof
\end{itemize}
Don't deduct marks if they omit the conclusion.
\end{markingScheme}

\end{solution}


%%%%%%%%%%%%%%%%%%%%%%%%%%%%%%%%%%%%%%%%%%%%%%%%%%%%%%%%%%%%%%%%%%%%%%%%%%%%%%
\item Any proper ternary tree is such that $e=2i+1$.
(You are not asked to prove this.)  Is it true for
general ternary trees? Prove it by an induction or an implication, or
disprove it by giving a counter-example.
%%%%%%%%%%%%%%%%%%%%%%%%%%%%%%%%%%%%%%%%%%%%%%%%%%%%%%%%%%%%%%%%%%%%%%%%%%%%%%
\begin{solution}
{\bf This property is false on general ternary trees}.
%
We disprove it by a counter example:
%
The following (binary) tree has at most three children per node hence
it is a {\em ternary} tree.
%
Moreover, it has one internal node ($i=1$), and
two external nodes ($e=2$), but $2i+1=3 \neq 2=i$.
\begin{center}
\Tree [ .1 2 3 ]
\end{center}
\end{solution}

\begin{markingScheme}
\begin{itemize}
\item $2$ marks if the counter-example is correct.
\end{itemize}
\end{markingScheme}


%%%%%%%%%%%%%%%%%%%%%%%%%%%%%%%%%%%%%%%%%%%%%%%%%%%%%%%%%%%%%%%%%%%%%%%%%%%%%%
\item Prove by an induction or an implication, or disprove by giving a
counter-example, that any proper ternary tree is such that
%
$$2h+1\leq e \leq 3^h.$$

Note that although the right inequality is true for any general ternary
tree (something you are not asked to prove), the left inequality
is {\bf false} on general ternary trees:
a tree with two nodes, a root and one child of the root, has $h=1$ and $e=1$.

%%%%%%%%%%%%%%%%%%%%%%%%%%%%%%%%%%%%%%%%%%%%%%%%%%%%%%%%%%%%%%%%%%%%%%%%%%%%%%
\begin{solution}
Both inequalities are true for any proper ternary tree $T$.
%
We prove it by implication, applying the equality $e=2i+1$ to the
double inequation $h\leq i \leq {3^h-1\over 2}$ to get
$2h+1\leq e \leq 3^h$.

\begin{markingScheme}
\begin{itemize}
\item $2$ marks for the implication, or if the student did the
an induction instead of an implication.
\end{itemize}
\end{markingScheme}

\end{solution}




%%%%%%%%%%%%%%%%%%%%%%%%%%%%%%%%%%%%%%%%%%%%%%%%%%%%%%%%%%%%%%%%%%%%%%%%%%%%%%


\end{enumerate}


\begin{hint}
When proving an induction, it is better to mention explicitly
the variable on which you base your induction;
the induction hypothesis (in function of this variable);
the base case of your hypothesis;
the inductive step;
the conclusion.
\end{hint}
