\begin{authorship}J{\'e}r{\'e}my Barbay\end{authorship}
\begin{usage}
Appeared in a simplified form in
Assignment 2, CS240, Winter term 2005, University of Waterloo, Canada
Assignment 3, CS240, Winter 2006 (School of CS, U of W). Average mark: 9.21/15
\end{usage}

A {\em ternary tree} is a tree such that each node has at most $3$ children.
A {\em proper} ternary tree is a tree such that each internal node has exactly $3$ children.
%
Given a ternary tree $T$, we define
  $i$ as its number of internal nodes,
  $e$ as its number of external nodes, and
  $h$ as its height.
%
For each of the following properties, prove or disprove it for both ternary
trees and proper ternary trees. Note that a property might be true for one type of tree but false for another.

\begin{INUTILE}
 \begin{markingScheme}
  $10$ points for each of the two first questions, and $5$ points for
  the third.
\end{markingScheme}
\end{INUTILE}

\begin{enumerate}

%%%%%%%%%%%%%%%%%%%%%%%%%%%%%%%%%%%%%%%%%%%%%%%%%%%%%%%%%%%%%%%%%%%%%%%%%%%%%%
\item $h\leq i \leq {3^h-1\over 2}$
%%%%%%%%%%%%%%%%%%%%%%%%%%%%%%%%%%%%%%%%%%%%%%%%%%%%%%%%%%%%%%%%%%%%%%%%%%%%%%
\begin{solution}

{\bf This property is true on general ternary trees}, we prove it by
induction on $h$. Note that this implies also that {\bf the property
is true on proper ternary trees}

\begin{itemize}

\item {\bf The induction is based on the variable} $h$, the height of
the tree.

\item {\bf The induction hypothesis is} $H(h)=$ ``For any proper
ternary tree of height at most $h$ and with $i$ internal nodes,
%
$h\leq i \leq {3^h-1\over2}$''.

\item {\bf The base case of the hypothesis is} $H(0)=$''For any proper
ternary tree of height $0$ and with $i$ internal nodes,
$0\leq{i}\leq {3^0-1\over2}=0$''.
%
As a ternary tree of height zero can have only one external node and
no internal nodes ($i=0$), $H(0)$ {\bf is verified}.

\item For any value of $h\in\{1,\ldots,\infty\}$, {\bf suppose that
$H(h-1)$ is true.}  The inductive step goes as follow:
%
For any ternary tree $T$ of height $h\geq1$, consider the children of
the root of $T$: there are at least one, and at most three, and each
is of height at most $h-1$: $H(h-1)$ apply to them.
%
One one hand, $T$ has at least one sub-tree and its number of internal
nodes is at least the number of internal nodes (by induction
hypothesis at least $h-1$) of the tallest of its subtrees plus one:
$i\geq h-1+1=h$.
%
On the other hand, the number of internal nodes in each subtree is at
most ${3^{h-1}-1\over 2}$, and $T$ has at most $3$ subtrees, hence its
number of internal nodes is at most
$1+3*{3^{h-1}-1\over 2}
= {3^h-3+2\over2}
= {3^h-1\over2}$.
%
Hence the final inequality $h\leq i \leq {3^{h}-1\over 2}$, {\bf which
proves that $H(h)$ is verified}.

\item {\bf The conclusion} is that for $h\in\{0,\ldots,\infty\}$, any
ternary tree of height $h$ has between $h$ and ${3^h-1}\over2$ internal nodes.

\end{itemize}

\begin{INUTILE}
  \begin{markingScheme}
    \begin{itemize}
    \item $1$ mark for the remark that they don't need to prove it for
      proper ternary trees (or for the long induction proof proving
      it);
    \item $2$ mark if among the five points of the induction none is
      missing, even if the induction is wrong;
    \item $7$ marks in general for the induction proof.
    \end{itemize}
  \end{markingScheme}
\end{INUTILE}
\end{solution}


%%%%%%%%%%%%%%%%%%%%%%%%%%%%%%%%%%%%%%%%%%%%%%%%%%%%%%%%%%%%%%%%%%%%%%%%%%%%%%
\item $e=2i+1$
%%%%%%%%%%%%%%%%%%%%%%%%%%%%%%%%%%%%%%%%%%%%%%%%%%%%%%%%%%%%%%%%%%%%%%%%%%%%%%

\begin{solution}
{\bf This property is false on general ternary trees}.
%
We disprove it by a counter example:
%
The following (binary) tree has at most three children per node hence
it is a {\em ternary} tree.
%
Moreover, it has one internal node ($i=1$), and
two external nodes ($e=2$), but $2i+1=3 \neq 2=i$.
\begin{center}
\Tree [ .1 2 3 ]
\end{center}

\medskip

%%%%%%%%%%%%%%%%%%%%%%%%%%%%%%%%%%%%%%%%%%%%%%%%%%%%%%%%%%%%%%%%%%%%%%%%%%%%%%
On the other hand, {\bf the property is true on proper ternary trees}.
%
We prove it by induction on the number $i$ of internal nodes:
\begin{itemize}

\item {\bf The induction is based on the variable} $i$, the number of
internal nodes.

\item {\bf The induction hypothesis is} $H(i)=$ ``each proper ternary
tree of $i$ internal nodes is such that $e=2i+1$''.

\item {\bf The base case of the hypothesis is} $H(0)=$''each proper
ternary tree of $0$ internal nodes is such that $e=2*0+1=1$''. As a
tree with no internal node can have only one external node ($e=1$),
{\bf $H(0)$ is verified}.

\item For any value of $i\in\{1,\ldots,\infty\}$, {\bf suppose that
$H(i-1)$ is true.}  The inductive step goes as follow:

For any proper ternary tree $T$ of $i$ internal nodes, note $e$ its
number of external nodes..
%
Obtain the proper ternary tree $T'$ by picking any (internal) node of
height $1$ in $T$, and replace it and its three children (all external
nodes) by a single external node.
%
$T'$ has $i'=i-1$ internal nodes, and two less external nodes than
$T$, as we removed three external nodes and added one: $T'$ has
$e'=e-2$ external nodes.

By the induction hypothesis $H(i-1)$, the number of external nodes of
$T'$ is $e'=2i'+1$, hence the number of external nodes of $T$ is
$e=e'+2=(2i'+1)+2=(2(i-1)+1)+2=2i+1$, {\bf which proves that $H(i)$ is
verified}.

\item {\bf The conclusion} is that for $i\in\{0,\ldots,\infty\}$, any
proper ternary tree of $i$ internal nodes has $e=2i+1$ external nodes.

\end{itemize}

\end{solution}

\begin{INUTILE}
  \begin{markingScheme}
    \begin{itemize}
    \item $1$ mark for the couter-example;
    \item $2$ mark if among the five points of the induction none is
      missing, even if the induction is wrong;
    \item $7$ marks in general for the induction proof.
    \end{itemize}
  \end{markingScheme}
\end{INUTILE}
%%%%%%%%%%%%%%%%%%%%%%%%%%%%%%%%%%%%%%%%%%%%%%%%%%%%%%%%%%%%%%%%%%%%%%%%%%%%%%
\item $2h+1\leq e \leq 3^h$
%%%%%%%%%%%%%%%%%%%%%%%%%%%%%%%%%%%%%%%%%%%%%%%%%%%%%%%%%%%%%%%%%%%%%%%%%%%%%%
\begin{solution}

The right inequality is true for any general ternary tree, but the
left equality is {\bf false} on general ternary trees.
%
We disprove it by a counter example:
%
The following (unary) tree is of height $h=2$, but has only one
external node ($e=1$), which contradicts the left inequality.
\begin{center}
\Tree [ .1 [ .2  3 ] ]
\end{center}
%
As the left inequality is not true, {\bf the double inequation is
false for general ternary trees}.  Note that we don't need to prove
that the right inequality is correct for any general tree, but if we
would the proof would be very similar to the proof of the whole
inequation for proper ternary trees.

\medskip

%%%%%%%%%%%%%%%%%%%%%%%%%%%%%%%%%%%%%%%%%%%%%%%%%%%%%%%%%%%%%%%%%%%%%%%%%%%%%%
On the other hand, {\bf both inequalities are true for any proper
ternary tree} $T$.
%
We prove can prove it either by implication, applying the equality
$e=2i+1$ to the double inequation $2h+1\leq e \leq 3^h$, or the long
way by induction on $h$:

\begin{itemize}

\item {\bf The induction is based on the variable} $h$, the height of the tree.

\item {\bf The induction hypothesis is} $H(h)=$ ``For any proper
ternary tree of height at most $h$ and with $e$ external nodes,
%
$2h+1\leq e \leq 3^h$''.

\item {\bf The base case of the hypothesis is} $H(0)=$''For any proper
ternary tree of height $0$ and with $e$ external nodes,
$1\leq{e}\leq3^0=1$''.
%
As a proper ternary tree of height zero can have only one external
node ($e=1$), $H(0)$ {\bf is verified}.

\item For any value of $h\in\{1,\ldots,\infty\}$, {\bf suppose that
$H(h-1)$ is true.}  The inductive step goes as follow:
%
For any proper ternary tree $T$ of height $h\geq1$, consider the three
sub-trees $T_1,T_2$ and $T_3$ of heights $h_1,h_2,h_3$ and respective
number of external nodes $e_1,e_2,e_3$, which roots are the children
of the root of $T$: the number of external nodes of $T$ is
$e=e_1+e_2+e_3$, and the height of $T$ is $h=1+\max\{h_1,h_2,h_3\}$.
%
By the induction hypothesis $H(h-1)$,
\begin{eqnarray*}
2h_1+1\leq e_1 \leq 3^{h_1};\\
2h_2+1\leq e_2 \leq 3^{h_2};\\
2h_3+1\leq e_3 \leq 3^{h_3}.
\end{eqnarray*}

On one hand
$$e
=     e_1+e_2+e_3
\leq  3^{h_1}+3^{h_2}+3^{h_3}
\leq  3*3^{\max\{h_1,h_2,h_3\}}
=     3^{h}.$$

On the other hand
$$e
=     e_1+e_2+e_3
\geq (2h_1+1) + (2h_2+1) + (2h_3+1)
\geq 2(\max\{h_1+h_2+h_3\}+1) +1
= 2h+1.$$
%
Hence the final inequality $2h+1\leq e \leq 3^{h}$, {\bf which proves
that $H(h)$ is verified}.

\item {\bf The conclusion} is that for $h\in\{0,\ldots,\infty\}$, any
proper ternary tree of height $h$ has between $2h+1$ and $3^h$ external nodes.


\end{itemize}

\begin{INUTILE}
  \begin{markingScheme}
    \begin{itemize}
    \item $2$ marks for the counter example,
    \item $3$ marks for the implication, or only $2$ points if the
      student did an induction and the induction if it is done totally
      correctly.
    \end{itemize}
  \end{markingScheme}
\end{INUTILE}
\end{solution}




%%%%%%%%%%%%%%%%%%%%%%%%%%%%%%%%%%%%%%%%%%%%%%%%%%%%%%%%%%%%%%%%%%%%%%%%%%%%%%


\end{enumerate}


\begin{hint}
  Note that an inequality of the form $a \le b \le c$ is considered to
  be two inequalities and thus each must be evaluated separately.
  %
  Each of the results can either be disproved by a counter-example or
  proved by induction. You may also prove a property by implication
  from other results asked in the problem.
  %
  No points will be given for circular reasoning: If you are asked to
  prove $A$ or $B$, remember that proving $(A\Rightarrow B)$ and
  $(B\Rightarrow A)$ proves $(A\Leftrightarrow B)$, but it does not
  prove $A$ nor does it prove $B$.
  %
  Note that when proving something by induction, it is a good idea to
  explicitly state the variable on which you base your induction, the
  induction hypothesis (as a function of this variable), the base case
  of your hypothesis, the inductive step, and the conclusion.
\end{hint}


\begin{codeExample}
  One way to structure your induction proof in \LaTeX\ is using the
  \verb+description+ environment, which is similar to the
  \verb+itemize+ environment but label each item in bold.

 \pagebreak[3]
  For instance, the following code:
\begin{verbatim}
\begin{description}
\item[Induction variable:] $n$ the number of nodes in the tree
\item[Induction hypothesis:] $H(n)=$ ``Any tree on less than $n$
  nodes has height at most $n+1$''.
\item[Base case:] $H(2)$ corresponds to the case where (...)
\item[Inductive step:] Supposed that $H(n)$ is true for some
  $n$.  We now show that $H(n+1)$ is true: (...)
\item[Conclusion:] By induction, $H(n)$ is true for all values
  of $n$ larger or equal to $2$.
\end{description}
\end{verbatim}
would give the following:
\begin{center}
  \fbox{\begin{minipage}[t]{.75\textwidth}
      \begin{description}
      \item[Induction variable:] $n$ the number of nodes in the tree
      \item[Induction hypothesis:] $H(n)=$ ``Any tree on less than $n$
        nodes has height at most $n+1$''.
      \item[Base case:] $H(2)$ corresponds to the case where (...)
      \item[Inductive step:] Supposed that $H(n)$ is true for some
        $n$.  We now show that $H(n+1)$ is true: (...)
      \item[Conclusion:] By induction, $H(n)$ is true for all values
        of $n$ larger or equal to $2$.
      \end{description}
    \end{minipage}}
  \end{center}
\end{codeExample}
