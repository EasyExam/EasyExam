\begin{authorship}
Goodrich \& Tamassia, problem C-2.31
\end{authorship}
\begin{usage}
\end{usage}

Show how to find the largest $k$ elements of a set of $n$
integers in $O(n+k\log n)$ time.

\begin{solution}
The idea is to use a PriorityQueue and to extract the maximum element $k$
times.  With the right implementation of PriorityQueue, this takes the
desired time.

Namely, use a MaxHeap to implement the priority queue.  We have seen in
class that given $n$ elements, we can build a MaxHeap in $O(n)$ time.
Then we have a valid heap for which we can extract the maximum in
$O(\log n)$ time.  We need to extract the maximum $k$ times in order to
get the $k$ largest elements, so finding the $k$ largest elements takes
$O(k\log n)$ time.  (In fact, a little less, since the number of elements
in the heap decreases, but $O(k\log n)$ is surely an upper bound.)
The total time for the algorithm therefore is $O(n+k\log n)$.
\end{solution}

\begin{markingScheme}
5 marks for ``building heap at once''.
5 marks for saying that this takes only O(n).
5 marks for saying ``extract max k times''.
5 marks for saying that this takes O(k log n).
\end{markingScheme}
