\begin{usage}
Assignment 2, Year 2002, Winter Term, School of Computer Science at the University of Waterloo (simplified version).\\
Assignment 2, Year 2005, Winter Term, School of Computer Science at the University of Waterloo.
\end{usage}
\begin{authorship}
(G\&T R-6.7)  
\end{authorship}

Would you use the adjacency list structure or the adjacency matrix
structure in each of the following cases?  
%
Justify your choice, or your lack of choice if the choice of the
data-structure does not matter too much.


\begin{enumerate}

\item The graph has $10,000$ vertices and $20,000$ edges, and it is 
important to use as little space as possible.
\begin{solution}
The adjacency list structure would take much less space than the
adjacency matrix for this ratio of edges to vertices, since the space
required is linear is the number of edges and vertices, where the
matrix is quadratic in the number of vertices. So the adjacency list
is the optimal choice under these conditions.
\end{solution}
\begin{markingScheme}
$2$ marks.
\end{markingScheme}


\item The graph has $10,000$ vertices and $20,000,000$ edges, and it is
important to use as little space as possible.
\begin{solution}
For this ratio of edges to vertices, the space required for each of
the two structures is roughly the same. 
%
The adjacency list will be slightly less, but the operations in
general will take much more time to perform than the adjacecy
matrix. 
%
Therefore, if we are only considering space, the adjacency list is the
best choice.
%
But, if we are also considering runtime as a less but still important
feature, then the extra space that the Matrix would take can be
sacrificed for the much quicker operations that can be performed.
\end{solution}
\begin{markingScheme}
$2$ marks
\end{markingScheme}

\item You need to answer the query {\tt areAdjacent} as fast as
possible, no matter how much space you use.
\begin{solution}
For the operation {\tt areAdjacent}, the adjacency matrix can be used
in constant time, where the adjacency list needs linear time on
average. 
%
So the adjacency matrix is the optimal choice for this condition.
\end{solution}
\begin{markingScheme}
$2$ marks
\end{markingScheme}

\end{enumerate}


