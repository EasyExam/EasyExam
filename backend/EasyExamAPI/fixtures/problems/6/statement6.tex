\begin{usage}
Assignment 1, Year 2005, Winter Term, School of Computer Science at the University of Waterloo. \\
Midterm, Fall 2005.
\end{usage}
\begin{authorship}
Text: unknown; solution: Anna Lubiw
\end{authorship}

\newcommand{\answerbox}{
\hspace*{\fill}
\fbox{\rule{0mm}{10mm}\rule{50mm}{0mm}}
}


Analyze the running time (in terms of $n$) for each of the following
code fragments using $\Theta$-notation.
%
\begin{spaceForAnswer}  Place your answer inside the box.\end{spaceForAnswer}%
%
You do not need to justify.
\begin{enumerate}

\item
\begin{verbatim}
sum := 0
for i from 1 to n*n*n do
   for j from 1 to i*i*i do
      sum := sum + 1;
\end{verbatim}

  \begin{spaceForAnswer}
    \answerbox
  \end{spaceForAnswer}

\begin{solution}
$$\sum_{i=1}^{n^3} \sum_{j=1}^{i^3} 1
= \sum_{i=1}^{n^3} i^3
\approx { (n^3)^{3+1} \over 3+1 }
\in \Theta(n^{12})$$
\end{solution}

\item
\begin{verbatim}
sum := 0
for i from 1 to 2*(lg n) do   // assume n is a power of 2
   for j from 1 to 4^i do     // 4^i means 4 to the power of i
      sum := sum + 1
\end{verbatim}

  \begin{spaceForAnswer}
    \answerbox
  \end{spaceForAnswer}
\begin{solution}
$$\sum_{i=1}^{2\lg n} \sum_{j=1}^{4^i} 1
= \sum_{i=1}^{2\lg n} 4^i
= {  4(4^{2\lg n} - 1)  \over  4-1 }
= {  4(2^{4\lg n} - 1)  \over 3}
= {  4(2^{\lg n^4}) - 4 \over 3}
= {  4n^4 - 4 \over 3}
\in \Theta(n^4)$$

Note that $\sum_{i=1}^{N} a^i = {a^{N+1} - 1 \over a-1} - 1 = { a(a^N -1) \over a-1}$

Note also that
$4^{2\lg n}
= (2^2)^{2\lg n}
= 2^{2*2\lg n}
= 2^{4\lg n}
= 2^{\lg( n^4 ) }
= n^4$.

\end{solution}

\item
\begin{verbatim}
sum := 0
for i from 1 to n*n do
   for j from 1 to i do
      sum := sum + 1
\end{verbatim}

  \begin{spaceForAnswer}
    \answerbox
  \end{spaceForAnswer}
\begin{solution}
$$\sum_{i=1}^{n^2} \sum_{j=1}^i 1 = \sum_{i=1}^{n^2} i = {n^2(n^2 + 1) \over 2} \in
\Theta(n^4)$$
\end{solution}

\item
\begin{verbatim}
for i from 1 to n*n do
   sum := n
   while (sum>0) do
      sum := sum - 1
\end{verbatim}

  \begin{spaceForAnswer}
    \answerbox
  \end{spaceForAnswer}

\begin{solution}
$$\sum_{i=1}^{n^2} \sum_{s=n}^1 1 = \sum_{i=1}^{n^2} n = n^3 \in \Theta(n^3)$$
\end{solution}

%\item
%\begin{verbatim}
%sum := 0
%s := 3*n
%while (s>0)
%   sum := sum + 1
%   if (s is even)
%      s := s/2 - 1;
%   else
%      s := s-1;
%\end{verbatim}

\end{enumerate}

\begin{markingScheme}
For each question, half of the marks for the justification, half for the result.
\end{markingScheme}

\begin{codeExample}
  To type your answers in \LaTeX, you might use the \verb+enumerate+
  environment (the solution showed here is bogus):
\begin{verbatim}
\begin{enumerate}
\item The complexity of the algorithm is $\sum_{i=0}^{n^{99}} 2^n$.
This is asymptotically of the order of $O(n^{99} \times 2^n).
\item (...)
\end{enumerate}
\end{verbatim}
\end{codeExample}
