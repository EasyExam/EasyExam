\begin{authorship} Anna Lubiw, from Weiss's book
\end{authorship}
\begin{usage}
Assignment 2, Year 2005, Winter Term, School of Computer Science at the University of Waterloo.
\end{usage}

A min-max heap is a data structure that supports both deleteMin and
deleteMax in $O(\log n)$ time per operation (though you won't show that
in this question).
%
The structure is a binary tree, with the same shape property as a
normal heap, but with the following ordering property, where ``value''
indicates the value stored at a node, and $p(x)$ indicates the parent
of node $x$: for any node $x$ at depth $2, 4, 6, \ldots$ $$
\rm{value}(p(p(x))) < \rm{value}(x) < \rm{value}(p(x))$$
%
and for any node $x$ at depth $3, 5, \ldots$
%
$$ \rm{value}(p(x)) < \rm{value}(x) < \rm{value}(p(p(x)))$$
%
and for nodes $x$ at depth 1 (i.e. the children of the root)
%
$$\rm{value}(p(x)) < \rm{value}(x)$$
%

\begin{enumerate}
\item Draw the tree corresponding to the following min-max heap:
6, 81, 87, 14, 17, 12, 28, 71, 25, 80, 20, 52, 78, 31, 42, 32, 59, 16.
(This is the correct ordering for a min-max heap.)

\begin{solution}
\begin{center}
\Tree [
.6
[.81
  [.14
    [.71
      32
      59
    ]
    [.25
      16
    ]
  ]
  [.17
    80
    20
  ]
]
[.87
  [.12
    52
    78
  ]
  [.28
    31
    42
  ]
]
]
\end{center}
\end{solution}
\begin{spaceForAnswer}\pagebreak\end{spaceForAnswer}

\item Describe how to find and remove the minimum element in a min-max
  heap.  Analyze the complexity of your methods.  (You do not need to
  prove correctness.)

\begin{solution}
  \begin{authorship}Jeremy Barbay\end{authorship}
  The minimum element is at the root. To remove it, exchange it with
  the last element, as in a normal heap.
%
  Then, going down the tree, check if each node respects the min max
  heap property relative to its depth, its two children and its four
  grand-children.
%
  When it does not, reorder the three nodes of the path so that they
  respect the condition and check the property downward.

  As the tree is of height $\lg n$ and as each reoredering requires a
  finite number of operations, the whole deletion requires at most
  $O(\lg n)$ operations.
\end{solution}



\begin{spaceForAnswer}\vfill\end{spaceForAnswer}

\item Give an $O(\log n)$ time algorithm to insert a new element
into a min-max heap.  Include a brief justification of the running time.
(You do not need to prove correctness.)

\begin{spaceForAnswer}\vfill\end{spaceForAnswer}

\begin{solution}
  \begin{authorship}Jeremy Barbay\end{authorship}
  Add the element to the first leaf available, as in a normal heap.

  Then, going up the path to the root, check if each node respects the
  min max heap property relative to its depth:
  \begin{center}
    if $x$ has value $a$, $p(x)$ has value $b$ and $p(p(x))$ has value
    $c$, \\
    then $a\in[\min(b,c),\max(b,c)]$.
  \end{center}
  When it does not, reorder the three nodes of the path so that they
  respect the condition and check the property upward.

  As the tree is of height $\lg n$ and as each reoredering requires a
  finite number of operations, the whole insertion requires at most
  $O(\lg n)$ operations.
\end{solution}

\end{enumerate}

\begin{spaceForAnswer}\pagebreak\end{spaceForAnswer}

\begin{markingScheme}
1 + 3 + 6
\end{markingScheme}
